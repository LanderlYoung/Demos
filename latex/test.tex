% hello.tex - Our first LaTeX example!
\documentclass{article}
\usepackage{framed}
\usepackage{pbox}


\usepackage{amsmath}

\usepackage{mathtools}


\begein{equasion}


\usepackage{color}

\begin{document}

\section{demo}
Hello World!

\section{makebox and mbox}
\makebox[0pt]{Some text} over this text
\makebox[15ex][s]{Censored text}\hspace{-15ex}\makebox[15ex][s]{X X X X X}
Text \makebox[2\width][r]{running away}

\section{framebox}
\makebox[\textwidth]{c e n t r a l} \par
\makebox[\textwidth][s]{s p r e a d} \par
\framebox[1.1\width]{Guess I’m framed now!} \par
\framebox[0.8\width][r]{Bummer, I am too wide} \par
\framebox[1cm][l]{never mind, so am I}


\section{framed}
\begin{framed}
This is an easy way to box text within a document!
\end{framed}

\section{raisebox}
\begin{framed}
\raisebox{0pt}[0pt][0pt]{\Large%
\textbf{Aaaa\raisebox{-0.3ex}{a}%
\raisebox{-0.7ex}{aa}%
\raisebox{-1.2ex}{r}%
\raisebox{-2.2ex}{g}%
\raisebox{-4.5ex}{h}
}
}
he shouted but not even the next
one in line noticed that something
terrible had happened to him.
\end{framed}


\section{minipage and parbox}
\noindent
\fbox{\parbox[b][4em][t]{0.33\textwidth}{Some \\ text} }
\fbox{\parbox[c][4em][s]{0.33\textwidth}{Some \vfill text} }
\fbox{\parbox[t][4em][c]{0.33\textwidth}{Some \\ text} }

\fbox{
\parbox{\textwidth}{
Some very long text...Some very long text...Some very long text...Some very long text...
}
}

\pbox[b]{5cm}{This is long text that will be wrapped once it reaches five
centimeters.}


\section{rules}
\rule{3mm}{.1pt}%
\rule[-1mm]{5mm}{1cm}%
\rule{3mm}{.1pt}%
\rule[1mm]{1cm}{5mm}%
\rule{3mm}{.1pt}

\section{struts}

\section{amsmath}

\begin{displaymath}
\forall x \in X, \quad \exists y \leq \epsilon
\end{displaymath}

\begin{displaymath}
k_{n+1} = n^22 + k_n^2 - k_{n-1}
\end{displaymath}

\begin{equation}
x = a_0 + \cfrac{1}{a_1
+ \cfrac{1}{a_2
+ \cfrac{1}{a_3 + \cfrac{1}{a_4} } } }
\end{equation}

\begin{equation}
\frac{
\begin{array}[b]{r}
\left( x_1 x_2 \right)\\
\times \left( x'_1 x'_2 \right)
\end{array}
}{
\left( y_1y_2y_3y_4 \right)
}
\end{equation}

\begin{displaymath}
z = \overbrace{
\underbrace{x}_\text{real} + i
\underbrace{y}_\text{imaginary}
}^\text{complex number}
\end{displaymath}




\long\def\dummy#1{#1}
\dummy{First paragraph\par Second paragraph}



\end{document}